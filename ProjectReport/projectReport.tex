\documentclass[11pt,twoside,letterpaper]{article}

\begin{document}
\title{Orbit Determination via Topocentric Angular Observations}
\author{Jacob Bailey, Gustavo Lee, Michael Lesnewski}
\date{October 23, 2018}
\maketitle

  \begin{abstract}
    In this work, a set of topocentric angular observations of a
    satellite's motion are used to determine the salient parameters of
    the satellie's orbit. Two different methods of orbit determination
    are herein examined: the methods of Gauss and Laplace. After
    discussion of the merits and pitfalls of these methods, we
    demonstrate the accuracy of the two by computing a best-fit orbit
    for the Tiangon-1 satellite.
  \end{abstract}

  \section {Introduction}
  \paragraph{}
  The determination of patterns of motion for celestial bodies is a
  surprisingly difficult problem, and one that has been oft studied
  through the history of celestial mechanics. Although the aim of the
  method is simple, it has been incredibly fruitful in its products
  that the rest of science has benefitted from. The struggle of
  rationalizing Tycho's observational data on the known planets led to
  Kepler's three laws, which are a fundamental piece of our
  understanding of the solar system. The determination of the orbit of
  Ceres as it passed the sun in 1801-1802 led Gauss to develop the
  method of least squares regression, which has seen considerable use
  in the last century to fit models to observational data in all
  branches of science.
  
  \section{Theory of Orbit Determination}
  At this point, the pattern should be pretty obvious. This is where
  I'll leave things for now.

  \subsection {This is a subsection}
    
  \paragraph{}
  We can separate the sections into subsections. We can also add
  paragraphs, as below.
    
  \paragraph{}
  Here's the paragraph, within the subsection. These markings should
  be the way we separate out all our different paragraphs and
  sections.

  \paragraph{}
  Obviously, you shouldn't directly edit the pdf file, as that is
  generated by LaTeX. Instead, edit the source file in this directory
  (projectReport.tex).



  
\end{document}
