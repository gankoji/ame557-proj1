\documentclass[11pt,twoside,letterpaper]{article}

\usepackage{amsmath}
\begin{document}
\title{Orbit Determination via Topocentric Angular Observations}
\author{Jacob Bailey, Gustavo Lee, Michael Lesnewski}
\date{October 16, 2018}
\maketitle

  \begin{abstract}
    In this work, a set of topocentric angular observations of a
    satellite's motion are used to determine the salient parameters of
    the satellie's orbit. Two different methods of orbit determination
    are herein examined: the methods of Gauss and Laplace. After
    discussion of the merits and pitfalls of these methods, we
    demonstrate the accuracy of the two by computing a best-fit orbit
    for the Tiangon-1 satellite.
  \end{abstract}

  \section {Introduction}
  \paragraph{}
  The determination of patterns of motion for celestial bodies is a
  surprisingly difficult problem, and one that has been oft studied
  through the history of celestial mechanics. Although the aim of the
  method is simple, it has been incredibly fruitful in its products
  that the rest of science has benefitted from. The struggle of
  rationalizing Tycho's observational data on the known planets led to
  Kepler's three laws, which are a fundamental piece of our
  understanding of the solar system. The determination of the orbit of
  Ceres as it passed the sun in 1801 led Gauss to develop the method
  of least squares regression, which has seen considerable use in the
  last century to fit models to observational data in all branches of
  science.

  
  \section{Theory of Orbit Determination}
  
  \subsection {Gauss' Method}
    
  \paragraph{}
  Gauss developed his method of orbit determination to solve a
  troubling problem: on January 1st 1801, Giuseppe Piazzi discovered
  Ceres and was able to track it for 40 days before it was lost in the
  glare of the sun. As it continued its solar orbit, the problem was
  to determine the orbital path, and predict the position at which it
  would again become observable. Gauss is credited with the
  predictions which allowed another astronomer, Franz Xaver von Zach,
  to observe the minor planet again on December 31st of the same year.

  \subsubsection {Observational Quantities}
  \paragraph{}
  Gauss' method centers on two key quantities: the observer's position
  vector in the equatorial coordinate system, and unit vectors along
  the direction of the observation. The latter is a result of the
  observational technology at the time: ranging methods such as radar
  were not available at the time, and thus the best observation one
  could achieve with the technology of the time (telescopes) was
  simply two angles describing the orientation of the observed
  object's line of sight vector. 

  \paragraph{}
  The equatorial position vector of the observer can be found as
  \cite{Wikipedia_2018}

  \begin{equation}
    \begin{split}
      R_n =
      \left[
        \frac{R_e}{\sqrt{(1 - (2f -f^2)sin^2\phi)}} + H_n
        \right]cos\phi_n(cos\theta_n\hat{I} + sin\theta_n\hat{J}) \\+
      \left[
        \frac{R_e(1 - f)^2}{\sqrt{(1 - (2f -f^2)sin^2\phi)}} + H_n
        \right]sin\phi_n\hat{K}
    \end{split}
  \end{equation}

  \(R_n\) is the observer's position vector (in Equatorial Coordinate System)
  
  \(R_e\) is the equatorial radius of the body (e.g., Earth's Re is 6,378 km)
  
  \(f\) is the oblateness (or flattening) of the body (e.g., Earth's f is 0.003353)
  
  \(\phi_n\) is the respective geodetic latitude
  
  \(\phi'_n\) is the respective geocentric latitude
  
  \(H_n\) is the respective altitude
  
  \(\theta_n\) is the respective local sidereal time

  \paragraph{}
  
  The observation unit vector can also be found (in the topocentric
  coordinate system) via the following  \cite{Wikipedia_2018}:

  $\hat{\rho}_n = cos\delta_ncos\alpha_n\hat{I} + cos\delta_nsin\alpha_n\hat{J} + sin\delta_n\hat{K}$

  \subsubsection{Gauss' Algorithm}
  Once we have in hand at least three line-of-sight observations and
  the observer's position vector in the equatorial coordinate frame at
  those times, we can determine the position and velocity vectors of
  the orbiting object (and thus the classical orbital elements).

  We begin with the relevant time intervals:
  \begin{equation}
    \tau_1 = t_1 - t_2
  \end{equation}
  \begin{equation}
    \tau_3 = t_3 - t_2
  \end{equation}
  \begin{equation}
    \tau = t_3 - t_1
  \end{equation}

  The next step is to find the common scalar product, $D_0$:
  \begin{equation}
    D_0 = \hat{\rho_1}\cdot(\hat{\rho_2}\times\hat{\rho_3})
  \end{equation}

  Followed by the matrix quantities, $D_{mn}$:
  \begin{equation}
    D_{11} = R_1\cdot(\hat{\rho_2}\times\hat{\rho_3})
    \quad D_{12} = R_1\cdot(\hat{\rho_1}\times\hat{\rho_3})
    \quad D_{13} = R_1\cdot(\hat{\rho_1}\times\hat{\rho_2})
  \end{equation}
  \begin{equation}
    D_{21} = R_2\cdot(\hat{\rho_2}\times\hat{\rho_3})
    \quad D_{22} = R_2\cdot(\hat{\rho_1}\times\hat{\rho_3})
    \quad D_{23} = R_2\cdot(\hat{\rho_1}\times\hat{\rho_2})
  \end{equation}
  \begin{equation}
    D_{31} = R_3\cdot(\hat{\rho_2}\times\hat{\rho_3})
    \quad D_{32} = R_3\cdot(\hat{\rho_1}\times\hat{\rho_3})
    \quad D_{33} = R_3\cdot(\hat{\rho_1}\times\hat{\rho_2})
  \end{equation}

  Using the just calculated quantities, we build three coefficients
  for the scalar position.

  \begin{equation}
    A = \frac{1}{D_0}\left( -D_{12}\frac{\tau_3}{\tau} + D_{22} -D_{32}\frac{\tau_1}{\tau}\right)
  \end{equation}

  \begin{equation}
    B = \frac{1}{6D_0}\left( D_{12}\left(\tau_3^2 - \tau^2 \right)\frac{\tau_3}{\tau}
    + D_{32}\left(\tau^2 - \tau_1^2\right)\frac{\tau_1}{\tau} \right)
  \end{equation}

  \begin{equation}
    E = R_2\cdot\hat{\rho_2}
  \end{equation}

  We will also need the squared magnitude of the second observer position vector
  \begin{equation}
    R_2^2 = R_2\cdot R_2
  \end{equation}

  Using the coefficients just built, we build a polynomial in the
  scalar distance of the observation. Here, $\mu$ is the gravitational
  parameter of the focal body of the orbit. 

  \begin{equation}
    a = -\left(A^2 + 2AE + R_2^2 \right)
  \end{equation}
  \begin{equation}
    b = -2\mu B\left(A+E \right)
  \end{equation}
  \begin{equation}
    c = -{\mu}^2B^2
  \end{equation}

  These quantities are now the coefficients in an 8th order polynomial
  in the scalar distance of the second observation, $r_2$.

  \begin{equation}
    r_2^8 + ar_2^6 + br_2^3 + c = 0
  \end{equation}

  This polynomial can be solved by any suitable root finding routine,
  such as the Newton-Rhapson method. We note that since this is a
  radial distance from the focal body of the orbit, the root must be
  real. In the event there are multiple real roots of the polynomial,
  other measurements or data must be used to disambiguate the solution.

  With the orbital distance of the body fixed for one of the
  observations, we can now discern the slant range of the object from
  the observer, $\rho_n$.

  \begin{equation}
    \rho_1 = \frac{1}{D_0}\left[
      \frac{6\left(D_{31}\frac{\tau_1}{\tau_3} + D_{21}\frac{\tau}{\tau_3}\right)r_2^3
        + \mu D_{31}\left(\tau^2 - \tau_1^2\right)\frac{\tau_1}{\tau_3}}{6r_2^3
        + \mu\left(\tau^2 - \tau_3^2\right)} - D_{11}\right]
  \end{equation}

  \begin{equation}
    \rho_2 = A + \frac{\mu B}{r_2^3}
  \end{equation}

  \begin{equation}
    \rho_3 = \frac{1}{D_0}\left[
      \frac{6\left(D_{13}\frac{\tau_3}{\tau_1} - D_{23}\frac{\tau}{\tau_1}\right)r_2^3
        + \mu D_{13}\left(\tau^2 - \tau_3^2\right)\frac{\tau_3}{\tau_1}}{6r_2^3
        + \mu\left(\tau^2 - \tau_1^2\right)} - D_{33}\right]    
  \end{equation}

  With the slant ranges in hand, we can now easily calculate the
  orbital position vectors of the observed body to its focal body,
  $R_n$.

  \begin{equation}
    \vec{r}_n = \vec{R}_n + \rho_n{\hat{\rho}}_n
  \end{equation}

  To find the velocity of the orbiting body, we rely on a series
  expansion of the orbital motion about the midpoint of the
  observations.

  \begin{equation}
    \vec{v}_2 = \frac{1}{f_1g_3 - f_3g_1}\left(-f_3\vec{r}_1 + f_1\vec{r}_3\right)
  \end{equation}

  Where the expansion terms are:

  \begin{equation}
    f_1 = 1 - \frac{1}{2}\frac{\mu}{r_2^3}\tau_1^2
  \end{equation}
  \begin{equation}
    f_3 = 1 - \frac{1}{2}\frac{\mu}{r_2^3}\tau_3^2
  \end{equation}
  \begin{equation}
    g_1 = \tau_1 - \frac{1}{6}\frac{\mu}{r_2^3}\tau_1^3
  \end{equation}
  \begin{equation}    
    g_3 = \tau_3 - \frac{1}{6}\frac{\mu}{r_2^3}\tau_3^3
  \end{equation}

  The orbital determination problem is now complete. Since we have
  assumed a Keplerian orbit, the entirety of the orbit is defined by
  the six components of the second position vector and its associated
  velocity, $\vec{r}_2$ and $\vec{v}_2$ \cite{kluever_2018}.

  \subsection{Laplace's Method}
  We now turn our attention to the second classical method of orbit
  determination, that of Laplace. 

  \bibliography{bibliography}{}
  \bibliographystyle{plain}
\end{document}
