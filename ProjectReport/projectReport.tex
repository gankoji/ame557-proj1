\documentclass[11pt,twoside,letterpaper]{article}

\usepackage{amsmath}
\begin{document}
\title{Orbit Determination via Topocentric Angular Observations}
\author{Jacob Bailey, Gustavo Lee, Michael Lesnewski}
\date{October 16, 2018}
\maketitle

  \begin{abstract}
    In this work, a set of topocentric angular observations of a
    satellite's motion are used to determine the salient parameters of
    the satellie's orbit. Two different methods of orbit determination
    are herein examined: the methods of Gauss and Laplace. After
    discussion of the merits and pitfalls of these methods, we
    demonstrate the accuracy of the two by computing a best-fit orbit
    for the Tiangon-1 satellite.
  \end{abstract}

  \section {Introduction}
  \paragraph{}
  The determination of patterns of motion for celestial bodies is a
  surprisingly difficult problem, and one that has been oft studied
  through the history of celestial mechanics. Although the aim of the
  method is simple, it has been incredibly fruitful in its products
  that the rest of science has benefitted from. The struggle of
  rationalizing Tycho's observational data on the known planets led to
  Kepler's three laws, which are a fundamental piece of our
  understanding of the solar system. The determination of the orbit of
  Ceres as it passed the sun in 1801 led Gauss to develop the method
  of least squares regression, which has seen considerable use in the
  last century to fit models to observational data in all branches of
  science.

  
  \section{Theory of Orbit Determination}
  
  \subsection {Gauss' Method}
    
  \paragraph{}
  Gauss developed his method of orbit determination to solve a
  troubling problem: on January 1st 1801, Giuseppe Piazzi discovered
  Ceres and was able to track it for 40 days before it was lost in the
  glare of the sun. As it continued its solar orbit, the problem was
  to determine the orbital path, and predict the position at which it
  would again become observable. Gauss is credited with the
  predictions which allowed another astronomer, Franz Xaver von Zach,
  to observe the minor planet again on December 31st of the same year.

  \paragraph{}
  The following treatment of Gauss' method is the same as described at
  \cite{Wikipedia_2018}, which was developed in \cite{curtis_2014}.

  \subsubsection {Observational Quantities}
  \paragraph{}
  Gauss' method centers on two key quantities: the observer's position
  vector in the equatorial coordinate system, and unit vectors along
  the direction of the observation. The latter is a result of the
  observational technology at the time: ranging methods such as radar
  were not available at the time, and thus the best observation one
  could achieve with the technology of the time (telescopes) was
  simply two angles describing the orientation of the observed
  object's line of sight vector. 

  \paragraph{}
  The equatorial position vector of the observer can be found as
 

  \begin{equation} \label{obsPos}
    \begin{split}
      R_n =
      \left[
        \frac{R_e}{\sqrt{(1 - (2f -f^2)sin^2\phi)}} + H_n
        \right]cos\phi_n(cos\theta_n\hat{I} + sin\theta_n\hat{J}) \\+
      \left[
        \frac{R_e(1 - f)^2}{\sqrt{(1 - (2f -f^2)sin^2\phi)}} + H_n
        \right]sin\phi_n\hat{K}
    \end{split}
  \end{equation}

  \(R_n\) is the observer's position vector (in Equatorial Coordinate System)
  
  \(R_e\) is the equatorial radius of the body (e.g., Earth's Re is 6,378 km)
  
  \(f\) is the oblateness (or flattening) of the body (e.g., Earth's f is 0.003353)
  
  \(\phi_n\) is the respective geodetic latitude
  
  \(\phi'_n\) is the respective geocentric latitude
  
  \(H_n\) is the respective altitude
  
  \(\theta_n\) is the respective local sidereal time

  \paragraph{}
  
  The observation unit vector can also be found (in the topocentric
  coordinate system) via the following  \cite{Wikipedia_2018}:

  \begin{equation}\label{losVec}
    \hat{\rho}_n = cos\delta_ncos\alpha_n\hat{I}
    + cos\delta_nsin\alpha_n\hat{J} + sin\delta_n\hat{K}
  \end{equation}

  \subsubsection{Gauss' Algorithm}
  Once we have in hand at least three line-of-sight observations and
  the observer's position vector in the equatorial coordinate frame at
  those times, we can determine the position and velocity vectors of
  the orbiting object (and thus the classical orbital elements).

  We begin with the relevant time intervals:
  \begin{equation}
    \tau_1 = t_1 - t_2
  \end{equation}
  \begin{equation}
    \tau_3 = t_3 - t_2
  \end{equation}
  \begin{equation}
    \tau = t_3 - t_1
  \end{equation}

  The next step is to find the common scalar product, $D_0$:
  \begin{equation}
    D_0 = \hat{\rho_1}\cdot(\hat{\rho_2}\times\hat{\rho_3})
  \end{equation}

  Followed by the matrix quantities, $D_{mn}$:
  \begin{equation}
    D_{11} = R_1\cdot(\hat{\rho_2}\times\hat{\rho_3})
    \quad D_{12} = R_1\cdot(\hat{\rho_1}\times\hat{\rho_3})
    \quad D_{13} = R_1\cdot(\hat{\rho_1}\times\hat{\rho_2})
  \end{equation}
  \begin{equation}
    D_{21} = R_2\cdot(\hat{\rho_2}\times\hat{\rho_3})
    \quad D_{22} = R_2\cdot(\hat{\rho_1}\times\hat{\rho_3})
    \quad D_{23} = R_2\cdot(\hat{\rho_1}\times\hat{\rho_2})
  \end{equation}
  \begin{equation}
    D_{31} = R_3\cdot(\hat{\rho_2}\times\hat{\rho_3})
    \quad D_{32} = R_3\cdot(\hat{\rho_1}\times\hat{\rho_3})
    \quad D_{33} = R_3\cdot(\hat{\rho_1}\times\hat{\rho_2})
  \end{equation}

  Using the just calculated quantities, we build three coefficients
  for the scalar position.

  \begin{equation}
    A = \frac{1}{D_0}\left( -D_{12}\frac{\tau_3}{\tau} + D_{22} -D_{32}\frac{\tau_1}{\tau}\right)
  \end{equation}

  \begin{equation}
    B = \frac{1}{6D_0}\left( D_{12}\left(\tau_3^2 - \tau^2 \right)\frac{\tau_3}{\tau}
    + D_{32}\left(\tau^2 - \tau_1^2\right)\frac{\tau_1}{\tau} \right)
  \end{equation}

  \begin{equation}
    E = R_2\cdot\hat{\rho_2}
  \end{equation}

  We will also need the squared magnitude of the second observer position vector
  \begin{equation}
    R_2^2 = R_2\cdot R_2
  \end{equation}

  Using the coefficients just built, we build a polynomial in the
  scalar distance of the observation. Here, $\mu$ is the gravitational
  parameter of the focal body of the orbit. 

  \begin{equation}
    a = -\left(A^2 + 2AE + R_2^2 \right)
  \end{equation}
  \begin{equation}
    b = -2\mu B\left(A+E \right)
  \end{equation}
  \begin{equation}
    c = -{\mu}^2B^2
  \end{equation}

  These quantities are now the coefficients in an 8th order polynomial
  in the scalar distance of the second observation, $r_2$.

  \begin{equation}
    r_2^8 + ar_2^6 + br_2^3 + c = 0
  \end{equation}

  This polynomial can be solved by any suitable root finding routine,
  such as the Newton-Rhapson method. We note that since this is a
  radial distance from the focal body of the orbit, the root must be
  real. In the event there are multiple real roots of the polynomial,
  other measurements or data must be used to disambiguate the solution.

  With the orbital distance of the body fixed for one of the
  observations, we can now discern the slant range of the object from
  the observer, $\rho_n$.

  \begin{equation}
    \rho_1 = \frac{1}{D_0}\left[
      \frac{6\left(D_{31}\frac{\tau_1}{\tau_3} + D_{21}\frac{\tau}{\tau_3}\right)r_2^3
        + \mu D_{31}\left(\tau^2 - \tau_1^2\right)\frac{\tau_1}{\tau_3}}{6r_2^3
        + \mu\left(\tau^2 - \tau_3^2\right)} - D_{11}\right]
  \end{equation}

  \begin{equation}
    \rho_2 = A + \frac{\mu B}{r_2^3}
  \end{equation}

  \begin{equation}
    \rho_3 = \frac{1}{D_0}\left[
      \frac{6\left(D_{13}\frac{\tau_3}{\tau_1} - D_{23}\frac{\tau}{\tau_1}\right)r_2^3
        + \mu D_{13}\left(\tau^2 - \tau_3^2\right)\frac{\tau_3}{\tau_1}}{6r_2^3
        + \mu\left(\tau^2 - \tau_1^2\right)} - D_{33}\right]    
  \end{equation}

  With the slant ranges in hand, we can now easily calculate the
  orbital position vectors of the observed body to its focal body,
  $R_n$.

  \begin{equation}
    \vec{r}_n = \vec{R}_n + \rho_n{\hat{\rho}}_n
  \end{equation}

  To find the velocity of the orbiting body, we rely on a series
  expansion of the orbital motion about the midpoint of the
  observations.

  \begin{equation}
    \vec{v}_2 = \frac{1}{f_1g_3 - f_3g_1}\left(-f_3\vec{r}_1 + f_1\vec{r}_3\right)
  \end{equation}

  Where the expansion terms are:

  \begin{equation}
    f_1 = 1 - \frac{1}{2}\frac{\mu}{r_2^3}\tau_1^2
  \end{equation}
  \begin{equation}
    f_3 = 1 - \frac{1}{2}\frac{\mu}{r_2^3}\tau_3^2
  \end{equation}
  \begin{equation}
    g_1 = \tau_1 - \frac{1}{6}\frac{\mu}{r_2^3}\tau_1^3
  \end{equation}
  \begin{equation}    
    g_3 = \tau_3 - \frac{1}{6}\frac{\mu}{r_2^3}\tau_3^3
  \end{equation}

  The orbital determination problem is now complete. Since we have
  assumed a Keplerian orbit, the entirety of the orbit is defined by
  the six components of the second position vector and its associated
  velocity, $\vec{r}_2$ and $\vec{v}_2$ \cite{kluever_2018}.

  \subsection{Laplace's Method}
  We now turn our attention to the second classical method of orbit
  determination, that of Simon Pierre Laplace. This treatment will
  closely follow the one presented in \cite{bate_mueller_white_2015}.

  As before with Gauss' method, we expect the observations to only
  contain angular quantities, such as right ascension and declination,
  in the topocentric coordinate system. Begin by expressing the
  observations as line of sight vectors in the topocentric coordinate
  system using equations \ref{losVec}. Additionally, we require the
  position of the observer at the observation epochs, found via
  equation \ref{obsPos}.

  \subsubsection{The Orbital Position Vector}
  Noting that the position vector of the orbiting body at each epoch
  can be written as

  \begin{equation} \label{focalPos}
    \vec{r} = \rho_n\hat{\rho}_n + \vec{R}_n
  \end{equation}

  with $\rho_n$ the (as yet undetermined) slant range from the observer
  to the body, we differentiate the position vector twice to arrive at
  a relation between the body's position and the assumed
  (Keplerian) form of its dynamics:

  \begin{equation} \label{posDiff}
    \hat{\rho}_2\ddot{\rho}_2 + 2\dot{\hat{\rho}}_2\dot{\rho_2}
    + \left( \ddot{\hat{\rho}}_2 + \frac{\mu}{r^3}\hat{\rho}_2 \right)\rho_2
    = -\left( \ddot{\vec{R}}_2 + \mu\frac{\vec{R}_2}{r^3} \right)
  \end{equation}

  Here, we note that $r^3$ is simply the magnitude of the observed
  body's distance from its orbital focus. Also of note is that the
  above equation is only used for the second observation. The first
  and third of the set will be used in the numerical differentiation
  process to determine the derivatives of the line of sight vector, as
  follows.

  \begin{equation} \label{rhoDot}
    \dot{\hat{\rho}}(t) =
    \frac{2t - t_2 - t_3}{(t_1 - t_2)(t_1 - t_3)}\hat{\rho}_1
      + \frac{2t - t_1 - t_3}{(t_2 - t_1)(t_2 - t_3)}\hat{\rho}_2
      + \frac{2t - t_1 - t_2}{(t_3 - t_1)(t_3 - t_2)}\hat{\rho}_3
  \end{equation}

  \begin{equation} \label{rhoDDot}
    \ddot{\hat{\rho}}(t) =
    \frac{2}{(t_1 - t_2)(t_1 - t_3)}\hat{\rho}_1
      + \frac{2}{(t_2 - t_1)(t_2 - t_3)}\hat{\rho}_2
      + \frac{2}{(t_3 - t_1)(t_3 - t_2)}\hat{\rho}_3
  \end{equation}

  Writing \ref{posDiff} for the central observation, and including the
  information from \ref{rhoDot} and \ref{rhoDDot}, we note that
  \ref{posDiff} is now a three component equation in four unknowns
  $\rho, \dot{\rho}, \ddot{\rho},$ and $r$. Taking the components of
  this equation and writing the system in matrix form, we can attempt
  a solution via Cramer's rule (with some additional elimination in
  the matrix):

  \begin{equation} \label{D}
    D = 2\begin{vmatrix}
    \hat{\rho}_I& \dot{\hat{\rho}}_I& \ddot{\hat{\rho}}_I \\
    \hat{\rho}_J& \dot{\hat{\rho}}_J& \ddot{\hat{\rho}}_J \\
    \hat{\rho}_K& \dot{\hat{\rho}}_K& \ddot{\hat{\rho}}_K \\
    \end{vmatrix}
  \end{equation}

  Applying Cramer's rule to equation \ref{posDiff}, we can see that

  \begin{equation} \label{slantRange1}
    D\rho = -\begin{vmatrix}
    \hat{\rho}_I& 2\dot{\hat{\rho}}_I& \ddot{R}_I
    + \mu R_I/r^3 \\
    \hat{\rho}_J& 2\dot{\hat{\rho}}_J& \ddot{R}_J
    + \mu R_J/r^3 \\
    \hat{\rho}_K& 2\dot{\hat{\rho}}_K& \ddot{R}_K
    + \mu R_K/r^3 \\
    \end{vmatrix}
  \end{equation}

  We can further simplify this to

  \begin{equation} \label{slantRange2}
    \rho = \frac{-2D_1}{D} - \frac{2\mu D_2}{f^3D} , D \neq 0
  \end{equation}

  With

  \begin{equation} \label{D1}
    D_1 = -2\begin{vmatrix}
    \hat{\rho}_I& \dot{\hat{\rho}}_I& \ddot{R}_I \\
    \hat{\rho}_J& \dot{\hat{\rho}}_J& \ddot{R}_J \\
    \hat{\rho}_K& \dot{\hat{\rho}}_K& \ddot{R}_K \\
    \end{vmatrix}
  \end{equation}

  \begin{equation} \label{D2}
    D_2 = -2\frac{\mu}{r^3}\begin{vmatrix}
    \hat{\rho}_I& \dot{\hat{\rho}}_I& R_I \\
    \hat{\rho}_J& \dot{\hat{\rho}}_J& R_J \\
    \hat{\rho}_K& \dot{\hat{\rho}}_K& R_K \\
    \end{vmatrix}
  \end{equation}

  This yields an expression for the slant range, which is dependent
  only on the still unknown magnitude of the orbiting body's focal
  position vector $r$. Dotting \ref{focalPos} with itself, we find

  \begin{equation} \label{focalMag}
    r^2 = \rho^2 + 2\rho \hat{\rho}\cdot\vec{R} + R^2
  \end{equation}

  Once \ref{focalMag} is solved, the resulting magnitude can be
  substituted to \ref{slantRange2} to find the slant range, and the
  position vector can be determined from \ref{focalPos}.

  \subsubsection{Determining the Velocity Vector}
  If we return to \ref{posDiff} and again apply Cramer's rule, we find
  that the velocity can also be expressed as a function of determinants.

  \begin{equation} \label{slantVelocity}
    D\dot{\rho} = -D_3 - \frac{\mu}{r^3}D_4
  \end{equation}

  With

  \begin{equation} \label{D3}
    D_3 = \begin{vmatrix}
    \hat{\rho}_I& \ddot{R}_I& \ddot{\hat{\rho}}_I \\
    \hat{\rho}_J& \ddot{R}_J& \ddot{\hat{\rho}}_J \\
    \hat{\rho}_K& \ddot{R}_K& \ddot{\hat{\rho}}_K \\
    \end{vmatrix}
  \end{equation}

  \begin{equation} \label{D4}
    D_4 = \begin{vmatrix}
    \hat{\rho}_I& R_I& \ddot{\hat{\rho}}_I \\
    \hat{\rho}_J& R_J& \ddot{\hat{\rho}}_J \\
    \hat{\rho}_K& R_K& \ddot{\hat{\rho}}_K \\
    \end{vmatrix}
  \end{equation}

  We can then solve for the time derivative of the slant range as

  \begin{equation} \label{slantDot}
    \dot{\rho} = -\frac{D_3}{D} - \frac{\mu}{r^3}\frac{D_4}{D}
  \end{equation}

  Finally, with the slant range and slant velocity solved, we can
  differentiate \ref{focalPos} and obtain the velocity vector:

  \begin{equation}\label{focalVel}
    \vec{v} = \dot{\vec{r}} =
    \dot{\rho}\hat{\rho} + \rho\dot{\hat{\rho}} + \dot{\vec{R}}
  \end{equation}

  With the position and velocity vectors defined in three space within
  a suitable coordinate system, the Keplerian orbit is fully
  defined. Classical elements of the orbit (semi-major axis,
  eccentricity, inclination, etc) can be found as in \cite{kluever_2018}.

  \section{Results}
  With the theory of orbital determination in place, we turn to the
  results. The two methods described above were used to transform a
  set of observations made of the chinese satellite Tiangong-1, while
  it was de-orbiting due to a hardware malfunction.

  The observations were taken in the topocentric coordinate system,
  and reported as 3-tuples of Julian date, right ascension, and
  declination. The observational data are shown in table
  \ref{observations}.

  \begin{table}[]
    \centering
    \begin{tabular}{|l|l|l|}
      \hline
      Julian Date (Days)& Right Ascension (Degrees)& Declination (Degrees)\\ \hline
      2458130.5830398300 & 351.0088027 & 28.5903906 \\ \hline
      2458130.5830409615 & 351.0800657 & 28.6857577 \\ \hline
      2458130.5830421210 & 351.1562914 & 28.7798921 \\ \hline
      2458130.5830433145 & 351.2333939 & 28.8757338 \\ \hline
      2458130.5830444675 & 351.3126571 & 28.9712327 \\ \hline
    \end{tabular}
    \caption{Observational Data of Tiangong-1}
    \label{observations}
  \end{table}

  We now present the results of the orbital determination methods. As
  a baseline, we also show TLE data for the satellite's orbit before
  it's demise, which was found at \cite{tiangong1-orbit}. The
  ``official'' AFSPC data has been removed from their website, as the
  satellite is no longer in orbit. The classical elements of the
  Keplerian orbit are compared between the TLE, Gauss' method, and
  Laplace's method in table \ref{resultsTable}.

  \begin{table}[]
    \centering
    \begin{tabular}{llll}
      & TLE & Gauss & Laplace \\ \cline{2-4} 
      \multicolumn{1}{l|}{Semi-Major Axis} & \multicolumn{1}{l|}{6518 km} & \multicolumn{1}{l|}{6728.155 km} & \multicolumn{1}{l|}{6587.1 km} \\ \cline{2-4} 
      \multicolumn{1}{l|}{Eccentricity} & \multicolumn{1}{l|}{0.0005983} & \multicolumn{1}{l|}{0.0111799} & \multicolumn{1}{l|}{0.0099} \\ \cline{2-4} 
      \multicolumn{1}{l|}{Inclination} & \multicolumn{1}{l|}{42.7393} & \multicolumn{1}{l|}{42.7647} & \multicolumn{1}{l|}{42.7393} \\ \cline{2-4} 
      \multicolumn{1}{l|}{Ascending Node} & \multicolumn{1}{l|}{196.1141} & \multicolumn{1}{l|}{345.335} & \multicolumn{1}{l|}{345.3201} \\ \cline{2-4} 
      \multicolumn{1}{l|}{Arg. of Periapsis} & \multicolumn{1}{l|}{335.0657} & \multicolumn{1}{l|}{66.102} & \multicolumn{1}{l|}{218.1973} \\ \cline{2-4} 
      \multicolumn{1}{l|}{True Anomaly} & \multicolumn{1}{l|}{25.0237 (MA)} & \multicolumn{1}{l|}{346.0159} & \multicolumn{1}{l|}{193.9572} \\ \cline{2-4} 
      \multicolumn{1}{l|}{Epoch (Julian)} & \multicolumn{1}{l|}{2458210.171586} & \multicolumn{1}{l|}{2458130.5830} & \multicolumn{1}{l|}{2458130.5830} \\ \cline{2-4} 
    \end{tabular}
    \caption{Orbital Determination Results}
    \label{resultsTable}
  \end{table}

  It is clear that while the general size and shape of the orbit's
  determined from the observations agree - to some extent - with the
  published TLE data, the orientation of the orbital plane with
  respect to the earth does not. This could be explained by the
  difference in epochs, as the latest TLE data was observed more than
  80 Julian days after our observation set was taken, and the
  satellite was de-orbiting. It could also be explained by the
  inadequacy of our observations, which spanned a very short arc of
  the satellite's orbit.

  Finally, we show results for the prediction phase. After the initial
  orbit determination was complete, we propagated the satellites
  position to two points in the future: 60 minutes after the
  observation epoch, and 80 days after the epoch. This process was
  repeated for both determination methods. The results of each
  propagation is shown in table \ref{propagationResults}.

  \begin{table}[]
    \centering
    \begin{tabular}{lllll}
      & \multicolumn{2}{l}{Gauss} & \multicolumn{2}{l}{Laplace} \\
      & 60 min & 80 d & 60 min & 80 d \\ \cline{2-5} 
      \multicolumn{1}{l|}{f} & \multicolumn{1}{l|}{-138.564} & \multicolumn{1}{l|}{161.813} & \multicolumn{1}{l|}{78.934} & \multicolumn{1}{l|}{-123.929} \\ \cline{2-5} 
      \multicolumn{1}{l|}{rx} & \multicolumn{1}{l|}{775.549} & \multicolumn{1}{l|}{-5346.675} & \multicolumn{1}{l|}{1811.145} & \multicolumn{1}{l|}{752.547} \\ \cline{2-5} 
      \multicolumn{1}{l|}{ry} & \multicolumn{1}{l|}{-5111.923} & \multicolumn{1}{l|}{-2430.357} & \multicolumn{1}{l|}{-4916.406} & \multicolumn{1}{l|}{4817.465} \\ \cline{2-5} 
      \multicolumn{1}{l|}{rz} & \multicolumn{1}{l|}{-4392.244} & \multicolumn{1}{l|}{-3426.826} & \multicolumn{1}{l|}{-3970.561} & \multicolumn{1}{l|}{4482.459} \\ \cline{2-5} 
    \end{tabular}
    \caption{Orbital Propagation Results}
    \label{propagationResults}
  \end{table}
  
  \section{Conclusion}
  Without a baseline with an epoch much closer to the observations
  contained here, it's difficult to comment on the accuracy of either
  orbital determination method with serious confidence. However, we
  can make a few small notes. It appears that Laplace's method
  produced a better estimate of the size/shape orbital parameters
  $a, e,$ and $i$, at least when compared to the TLE data available.

  The propagation results also show considerable variability between
  the two methods. This is likely an artifact of the considerable
  disagreement between the method's results for orbital
  orientation. The magnitude of the difference in predicted position
  between the two methods is roughly 1200 km at the 60 minute
  prediction point, but that grows significantly to 12,000 km at the
  80 day mark. Given that this is roughly the length of the major axis
  of the orbit, the two predictions are clearly out of phase, and
  differences in predicted mean motion have caused divergence in the
  solution.

  If we were to depend on these results for a real world application,
  such as predicting the next patch of observability for the object,
  we would be hopelessly lost when turning our telescopes to the
  sky. Ideally, a significantly larger number of observations would be
  available, and Gauss' least squares methods could be used to refine
  the estimate of the orbit.   
  
  \bibliography{bibliography}{}
  \bibliographystyle{plain}
\end{document}
